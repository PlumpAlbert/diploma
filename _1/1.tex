\section{Литературный и патентный обзор постановки подобных задач.
Анализ стандартных средств и существующих способов решения задачи}

В настоящее время автомобильный парк страны пополняется автотранспортными
средствами новой конструкции, совершенствуется структура подвижного состава,
увеличивается численность дизельного парка, растет число транспортных средств
большой грузоподъемности. Темпы роста объемов перевозок и численность подвижного
состава растут и нуждаются в увеличении численности предприятий технического
обслуживания автотранспорта. Это вызывает необходимость исследования пути
ускорения научно-технического прогресса в отрасли, определить рациональные формы
и направления развития производственно-технической базы.

Для поддержания автомобилей в технически исправном состоянии необходимо
своевременное прохождение технического обслуживания ТО-1. Для этого создают
специальные пункты, в которых имеется всё необходимое оборудование для их
диагностики и ремонта. Диагностика осуществляется на основании государственного
стандарта ГОСТ 33997-2016 [\ref{ref:ГОСТ-требования-то}], в котором указаны
требования безопасности к техническому состоянию и методы проверки.

Из-за повышающегося количества автотранспортных средств загруженность пунктов
технического обслуживания стала велика. Существует несколько вариантов решения
данной проблемы. Возможно увеличение числа пунктов технического обслуживания.
Либо увеличить пропускную способность таких пунктов, за счет совершенствования
программы технического обслуживания ТО−1, улучшая ее технологичность,
качественность и респектабельность.

Нормативно-техническая документация (НТД), являющаяся одной из главных
составляющих системы технической эксплуатации машин, в том числе и грузовых
автомобилей, рассредоточена во многих публикациях. Определенные компоненты НТД,
в частности, по техническому обслуживанию (ТО) и техническому диагностированию
(ТД) грузовых автомобилей, постоянно развиваются и корректируются.
В силу указанных особенностей оперирование такой документацией,
включая подбор и систематизацию ее обновленных компонентов, представляет
значительные сложности, из-за чего на практике специалисты тратят много времени
на оперирование НТД и нередко используют устаревшие или неполные комплекты
документации. Это влечет к возможному снижению качества обслуживания
автотранспорта, что нередко случается на практике.
Одним из путей устранения указанных сложностей и совершенствования организации
ТО грузовых автомобилей является применение компьютерных средств для
оперирования НТД в процессе непосредственного выполнения операций обслуживания.
Однако в данной сфере отсутствуют доступные на практике приемы и методы
использования таких средств. Поэтому разработка приемов и методов ТО с
использованием указанных средств является одной из актуальных задач современной
инженерной науки.

Выполнение операций ТО может быть усовершенствовано применением соответствующих
технологических карт (ТК). Однако они разработаны лишь для некоторых моделей
автомобилей и не получили широкого практического применения.
Операции ТД весьма эффективны в процессе ТО. Однако из-за широкого спектра их
применимости в технической эксплуатации машин они пока не получили тесной
корреляции с операциями ТО, т.е. явно не вписаны в процессы ТО.

Еще одним приемом усовершенствования ТО является реализация метода
прогнозирования параметров состояния агрегатов и узлов автомобиля по результатам
ТД. Из-за значительной трудоемкости и сложности вычислений по прогнозированию,
отсутствия систематизированных данных по реализации, метод также не получил
практического применения.

\subsection{Примеры существующих программных продуктов на рынке}
1C:Предприятие. Автосервис.

\textquote{1С:Автосервис 8} – это отраслевое специализированное решение, предназначенное
для автоматизации управления и учета в автосервисах, станциях технического
обслуживания и автомойках.

Конфигурация \textquote{Автосервис} разработана на основе типовой конфигурации
\textquote{Управление нашей фирмой}, редакции 1.6 системы программ
\textquote{1С:Предприятие 8} с сохранением всех возможностей и механизмов
типового решения. Учитывает специфику предприятий авто бизнеса и обеспечивает
следующие возможности [\ref{ref:1С-автосервис}]:
\begin{itemize}
	\item ведение базы клиентов с регистрацией и хранением всей важной
		информации;
	\item тотальный контроль всех контактов с клиентами: входящие и
		исходящие звонки, электронные письма, встречи и прочее;
	\item предварительная запись на ремонт;
	\item анализ клиентской базы;
	\item использование справочников: модели автомобилей, нормочасы, виды
		ремонта и другие;
	\item регистрация и хранение номенклатуры товаров и услуг;
	\item учет движения денежных средств в кассе и на банковских счетах;
	\item учет рабочего времени сотрудников и расчет заработной платы;
	\item статистика, отчеты и другие показатели;
	\item облачное решение обеспечивает доступ к системе через интернет из любых
		браузеров.
	\\[\baselineskip]
\end{itemize}


iDirector

iDirector представляет собой современное узкоспециализированное решение для
\textquote{продвинутых} автосервисов. Сочетает в себе легкий понятный интерфейс
и мощные модули. Также имеет облачную версию, доступную даже с мобильных
устройств.

Помимо стандартных возможностей, таких как: управление клиентской базой, ведение
заказов, учет склада, управление персоналом, имеется и ряд дополнительных
возможностей [\ref{ref:iDirector}]:
\begin{itemize}
	\item интеграция с сайтом -– позволяет установить форму для оформления
		заказов на сайт автосервиса, а также добавить онлайн-консультанта;
	\item таймлайн -- сравнение изображения с камер наблюдения и сопоставление с
		фактическим занесением заказа в систему;
	\item резервирование –- автоматическое регулярное резервирование всех
		данных, и возможность восстановления системы к определенной дате.
	\\[\baselineskip]
\end{itemize}


\textquote{АвтоДилер} с модулем \textquote{Сервис}

Система «АвтоДилер» -- это специализированное программное обеспечение для
автобизнеса.

Система предназначена для автоматизации учета, планирования и анализа работы
любых предприятий: крупных и мелких автомастерских, автосалонов, магазинов
автозапчастей, автомоек, шиномонтажных мастерских и станций замены масла,
автостраховщиков.

Модуль \textquote{Сервис} предназначен для автоматизации учета работ в автосервисах и на
станциях технического обслуживания автомобилей. Система позволяет значительно
сократить время на оформление документов. При повторном обращении клиента, у
пользователя имеется вся история взаимоотношений с ним и не потребуется
\textquote{ползать} по многотомным архивам сервиса, для восстановления картины
ремонта автомобиля.

В программном комплексе существует возможность оформления необходимых документов
как для клиента, так и для внутренних операций.

Решение также имеет возможность вести учет и создавать отчеты по выработке.

Из минусов стоит выделить отсутствие облачной версии, что сегодня является
довольно весомым недостатком.
