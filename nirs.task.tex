% vim:ft=tex:ts=4:sw=0
% Author: Plump Albert (plumpalbert@gmail.com)

\begin{center}

\uppercase{
    Липецкий Государственный Технический Университет

    Факультет автоматизации и информатики

    Кафедра автоматизированных систем управления
    \\[\baselineskip]
    Задание на производственную практику
}

\end{center}

\noindent Студенту Федину Матвею Сергеевичу группы ПИ-18

\noindent Направление (специальность) 09.03.04 \textquote{Программная инженерия}

\noindent\textbf{Изучить:}
\begin{enumerate}
    \item Структуру, деятельность (бизнес-процессы), задачи предприятия и
        подразделения, в котором проходит практика.
    \item Используемые на предприятии и в подразделении средства
        информатизации и автоматизации: существующую информационную или
        автоматизированную систему (системы); аппаратные и программные
        средства, информационное обеспечение, математические методы, модели
        и алгоритмы.
    \item Действующий порядок разработки и использования средств
        информатизации и автоматизации.
\end{enumerate}

\noindent\textbf{Разработать решение} конкретной задачи по информатизации или
автоматизации:
\begin{enumerate}
    \item Изучить необходимые источники, существующие варианты решения.
    \item Собрать на предприятии реальные данные для решения задачи.
    \item Разработать математические методы, модели и алгоритмы;
        информационную базу; аппаратное и программное обеспечение.
\end{enumerate}
\vfill
