\documentclass[../nirs.tex]{subfiles}

\begin{document}
\section{Литературный и патентный обзор постановки подобных задач.
Анализ стандартных средств и существующих способов решения задачи}

В наши дни очень перспективным и стремительно развивающимся видом предоставления
услуг являются грузовые перевозки.
Уже стало обыденностью заказывать продукты с доставкой на дом, несколько лет
успешно функционируют интернет-магазины одежды и техники.
Из-за постоянно возрастающих человеческих потребностей на рынок выходят новые
фирмы, деятельность которых заключается в доставке грузов.

Одной из первоочередных задач, решаемых на предприятиях, является функция
учета хозяйственной деятельности.
Ведение документооборота на бумажных носителях не позволяют вести оперативного
учета работы организации и затрудняет доступ к информации, существенно замедляя
ее анализ.
На современном уровне развития информационных технологий данные проблемы
разрешимы с внедрением автоматизированной информационной системы.

Основной целью автоматизации является повышение качества бизнес-процесса и
снижение его стоимости.
На сегодняшний день существует огромный выбор программных продуктов, позволяющих
автоматизировать функцию учета деятельности на предприятии.
Однако большая часть из них ориентирована на широкий круг потребителей, что
делает эти системы не пригодными для ведения учета на предприятиях со
специфическими потребностями к учету хозяйственной деятельности,
в частности, на малых предприятиях, занимающихся перевозкой грузов.
Большинство решений по учету хозяйственной деятельности нацелено на крупные
компании.

Так, компания ТЭК ИНСОФТ разработало автоматизированную систему учета работы
автотранспорта и дорожно-строительной техники \textquote{АТЦ}. Это специализированное
программно-аппаратное решение, предназначенное для транспортных предприятий и
транспортных подразделений предприятий [1].
Основными возможностями данного программного комплекса являются полное
соответствие транспортному законодательству РФ и возможность создавать
исчерпывающие отчеты.
Внедрение данной системы позволит заметно ускорить работу диспетчера,
выпускающего технику на линию.
Комплекс позволяет автоматизировать процесс контроля за эксплуатацией
автотранспорта на основе обработанной информации по путевым листам, картам
технического состояния и заявкам, а так же для учет расхода ГСМ.
В состав \textquote{АТЦ} входят следующие автоматизированные рабочие места:

\begin{itemize}
	\item АРМ диспетчера;
	\item АРМ техника ГСМ;
	\item АРМ механика;
	\item АРМ бухгалтера и экономиста;
	\item АРМ начальника;
	\item АРМ охранника и механика КПП.
\end{itemize}

Другим примером готового решения является программный комплекс для управления
автотранспортным хозяйством \textquote{АвтоПлан} компании АвтоПлан. Программный
комплекс состоит из 9 взаимосвязанных модулей:
\begin{enumerate}
	\item Журнал механика по выпуску транспорта - позволяет фиксировать
		выезд/заезд транспорта, замечания к его техническому состоянию.
	\item Разнарядка - позволяет простейшим образом создавать план выпуска
		техники и печатать путевые листы.
	\item Обработка путевых листов - контроль эксплуатации транспорта,
		расходов на него, контрольные и управленческие отчеты, печать актов
		выполненных работ и многое другое.
	\item Табель - контрольные и управленческие отчеты по персоналу во всех
		возможных разрезах.
	\item Склад - приход-расход ТМЦ, распределение со склада, контроль расхода
		запчастей.
	\item Журнал ремонта транспорта - организация и оперативный контроль
		ремонта транспорта, регламентных ТО, учет шин и аккумуляторов.
	\item Заявочный модуль - устанавливается на стороне постоянных клиентов,
		позволяя автоматизировать создание заявок.
	\item Управленческий модуль - позволяет собрать в одном месте всю
		необходимую информацию, упростить формирование отчетов.
	\item Модуль водителя - позволяет поддерживать обратную связь с
		диспетчером.
\end{enumerate}

Они обеспечивают взаимодействие, контроль и управление всеми службами АТП, с
максимально возможной автоматизацией и максимальным исключением человеческого
фактора из управления [2].
Программный комплекс устроен таким образом, чтобы максимально разгрузить
персонал, обеспечить надежность работы и безопасность данных.
Поддерживается интеграция с GPS/ГЛОНАСС оборудованием, а также с
\textquote{1С:Бухгалтерия}.
\end{document}
