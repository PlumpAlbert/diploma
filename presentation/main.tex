\documentclass{beamer}
\usepackage{fontspec}
\usepackage[russian]{babel}
\usepackage{csquotes}
\usepackage{gost-table}

%%% beamer
\setmainfont{Times New Roman}
\usefonttheme{serif}
\setbeamertemplate{frametitle}[default][center]
% \usetheme{Singapore}
%%%

\begin{document}
\begin{frame}
    \begin{center}
        Липецкий государственный технический университет\\
        Кафедра автоматизированных систем управления
        \vfill
        ОТЧЕТ\\
        по преддипломной практике в ООО \textquote{Fusionsoft}\\
        по направлению 09.03.04 \textquote{Программная инженерия}\\
        \textquote{Разработка информационной системы поддержки технического
        обслуживания автомобильного парка}
    \end{center}
    \vfill
    \begin{tabularx}{\textwidth}{LR}
        Студент & Федин М.С. \\
        Группа ПИ-18 & \\
        Руководитель & \\
        к.т.н., доцент & Назаркин О.А.
    \end{tabularx}
\end{frame}

\begin{frame}
    {Характеристика объекта автоматизации или предметной области}
    Объектом управления разрабатываемой системы являются процессы учета и
    анализа деятельности подразделения организации, занимающегося обслуживанием
    автомобильной техники.

    Его деятельность нацелена на поддержание парка автомобилей в технически
    исправном состоянии.

\end{frame}

\begin{frame}
	{Постановка задачи. Цели, критерии оценки и ограничения}
    В процессе эксплуатации автомобиля его составные части изнашиваются, что
    приводит к снижению эффективности или даже поломке.
    \\[\baselineskip]

    Существует два способа поддержания работоспособности автомобиля:
    \begin{itemize}
        \item техническое обслуживание (ТО);
        \item ремонт (Р).
    \end{itemize}
\end{frame}

\begin{frame}
	{Постановка задачи. Цели, критерии оценки и ограничения}
Основные задачи, которые должна решать разрабатываемая система:
\begin{itemize}
	\item учет ресурсов, необходимых для проведения ТО;
	\item учет текущего состояния автомобилей;
	\item учет проведения сервисных работ;
	\item календарное планирование проведения ТО.
\end{itemize}
\end{frame}

\begin{frame}
	{Основные понятия и процессы, их свойства и закономерности}
\end{frame}

\begin{frame}
	{ER-диаграмма предметной области}
\end{frame}

\begin{frame}
	{Теоретические математические модели}
\end{frame}

\begin{frame}
	{Эмпирические математические модели}
\end{frame}

\begin{frame}
	{Логическая модель данных}
\end{frame}

\begin{frame}
	{Физическая модель данных}
\end{frame}

\begin{frame}
	{Описание источников информации, входных сигналов и документов}
\end{frame}

\begin{frame}
	{Описание выходной информации: сигналов, документов и видеокадров}
\end{frame}
\end{document}
