\documentclass[../nirs.tex]{subfiles}
\usepackage{rotating}

\begin{document}
\section{Выявление основных понятий и процессов, их свойств и закономерностей.
Построение ER-диаграммы предметной области}

\subsection{Основные понятия}
Путевой лист -- это документ для учёта и контроля работы водителя и
транспортного средства (п. 14 ст. 2 Федерального закона от 08.11.2007 № 259-ФЗ).
В нём прописывается маршрут и техническое состояние машины, информация о
проведённом медосмотре водителя и пр. Путевые листы нужны, чтобы обосновать
необходимость аренды или лизинга, а также подтвердить расходы, связанные с
использованием транспортных средств. Путевые листы составляют индивидуальные
предприниматели и организации всех форм собственности, которые используют
транспорт в своей деятельности или для собственных нужд [3].

Горюче-смазочные материалы (ГСМ) -- общее обозначение широкого спектра веществ,
обеспечивающих бесперебойную работу двигателей внутреннего сгорания и различных
технических узлов. Они включают смазочные материалы, горючее и технические
жидкости [4].

Техническое обслуживание -- это комплекс организационно-технических мероприятий
и работ, производимых на объекте и направленных на поддержание в рабочем или
исправном состоянии оборудования (программного обеспечения) технических систем в
процессе их использования по назначению, с целью повышения надежности и
эффективности их работы [5].

\subsection{Основные процессы}
Заключение договора -- во время которого рассматривается заказ на оказание
транспортных услуг, определяется необходимое для его выполнения количество
единиц техники, устанавливается цена за оказание транспортных услуг, а также
заключение договора с клиентом.

Выпуск техники -- процесс составления разнарядки, в соответствии с заявками
клиентов, выдача путевых листов водителям автотранспорта, предрейсовый контроль
технического состояния транспорта, выдача ГСМ выпущенной на линию техники.

Сервис и ремонт -- отвечает за календарное планирование графика прохождения
технического обслуживания транспортной техники, передача ее в сервис и
возвращение из него.

Документооборот -- процесс обработки путевых листов, с целью списания
горюче-смазочных средств, а также расчета работы техники за смену.

\subsection{Построение ER-диаграммы предметной области}
Классовая диаграмма разрабатываемой системы представлена на рисунке
\ref{fig:2_1_3_er_diagram}.

\begin{sidewaysfigure}[htbp]
	\centering
	\includegraphics[keepaspectratio,width=0.73\textwidth]{./images/2_1_3_er-diagram.png}
	\caption{ER-диаграмма системы}
	\label{fig:2_1_3_er_diagram}
\end{sidewaysfigure}

\end{document}
