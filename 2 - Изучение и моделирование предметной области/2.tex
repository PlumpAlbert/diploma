\documentclass[../nirs.tex]{subfiles}

\begin{document}
\section{Теоретическое изучение предметной области. Построение теоретических
математических моделей}

В соответствии с приказом Минтранса России от 11.09.2020 № 368 \textquote{%
	Об утверждении обязательных реквизитов и порядка заполнения путевых листов%
} [6]:

\subsection{Обязательные реквизиты путевого листа}
\begin{enumerate}
	\item Путевой лист должен содержать следующие обязательные реквизиты:

	\begin{enumerate}
		\item наименование и номер путевого листа;
		\item сведения о сроке действия путевого листа;
		\item сведения о собственнике (владельце) транспортного средства;
		\item сведения о транспортном средстве;
		\item сведения о водителе;
		\item сведения о перевозке.
	\end{enumerate}

	\item Сведения о сроке действия путевого листа включают дату (число, месяц,
		год), в течение которой путевой лист может быть использован, а в случае
		если путевой лист оформляется более чем на один день - даты (число,
		месяц, год) начала и окончания срока, в течение которого путевой лист
		может быть использован.
	\item Сведения о собственнике (владельце) транспортного средства включают:

	\begin{enumerate}
		\item для юридического лица - наименование, организационно-правовую
			форму, местонахождение, номер телефона, основной государственный
			регистрационный номер юридического лица;
		\item для индивидуального предпринимателя - фамилию, имя, отчество (при
			наличии), почтовый адрес, номер телефона, основной государственный
			регистрационный номер индивидуального предпринимателя.
	\end{enumerate}

	\item Сведения о транспортном средстве включают:

	\begin{enumerate}
		\item тип транспортного средства, марку и модель транспортного средства,
			а в случае если транспортное средство используется с прицепом
			(полуприцепом), кроме того - марку и модель прицепа (полуприцепа);
		\item государственный регистрационный номер транспортного средства, а в
			случае если транспортное средство используется с прицепом
			(полуприцепом), его регистрационный номер, и/или инвентарный номер
			(для троллейбусов и трамваев);
		\item показания одометра (полные километры пробега) при выезде
			транспортного средства с парковки (парковочного места),
			предназначенной для стоянки данного транспортного средства по
			возвращении из рейса и окончании смены (рабочего дня) водителя
			транспортного средства (далее - парковка), а также при заезде
			транспортного средства на парковку по окончании смены (рабочего
			дня);
		\item дату (число, месяц, год) и время (часы, минуты) проведения
			предрейсового или предсменного контроля технического состояния
			транспортного средства (если обязательность его проведения
			предусмотрена законодательством Российской Федерации);
		\item дату (число, месяц, год) и время (часы, минуты) выпуска
			транспортного средства на линию и его возвращения.
	\end{enumerate}

	\item Сведения о водителе включают:

	\begin{enumerate}
		\item фамилию, имя, отчество (при наличии);
		\item дату (число, месяц, год) и время (часы, минуты) проведения
			предрейсового и послерейсового медицинского осмотра водителя (если
			обязательность проведения послерейсового медицинского осмотра
			водителя предусмотрена законодательством Российской Федерации).
	\end{enumerate}

	\item Сведения о перевозке включают информацию о видах сообщения и видах
		перевозок.
	\item На путевом листе допускается размещение дополнительных реквизитов,
		учитывающих особенности осуществления деятельности, связанной с
		перевозкой грузов, пассажиров и багажа автомобильным транспортом или
		городским наземным электрическим транспортом.
\end{enumerate}

\subsection{Порядок заполнения путевого листа}
\begin{enumerate}
	\setcounter{enumi}{7}
	\item Путевой лист оформляется на каждое транспортное средство,
		эксплуатируемое юридическим лицом и (или) индивидуальным
		предпринимателем.
	\item Путевой лист оформляется до начала выполнения рейса, если длительность
		рейса водителя транспортного средства превышает продолжительность смены
		(рабочего дня), или до начала первого рейса, если в течение смены
		(рабочего дня) водитель транспортного средства совершает один или
		несколько рейсов.
	\item Если в течение срока действия путевого листа транспортное средство
		используется посменно несколькими водителями, то допускается оформление
		на одно транспортное средство нескольких путевых листов раздельно на
		каждого водителя.
	\item В наименовании путевого листа указывается тип транспортного средства,
		на которое оформляется путевой лист. Номер путевого листа указывается в
		заголовочной части в хронологическом порядке в соответствии с принятой
		владельцем транспортного средства системой нумерации.
	\item Даты, время и показания одометра при выезде транспортного средства с
		парковки и его заезде на парковку проставляются уполномоченными лицами,
		назначаемыми решением руководителя юридического лица или индивидуального
		предпринимателя, и заверяются их подписями с указанием фамилий и
		инициалов, за исключением случаев, когда индивидуальный предприниматель
		совмещает обязанности водителя.
	\item Даты, время и показания одометра при выезде транспортного средства с
		парковки и его заезде на парковку проставляются индивидуальным
		предпринимателем в случае, если указанный предприниматель совмещает
		обязанности водителя.
	\item В случае оформления на одно транспортное средство нескольких путевых
		листов раздельно на каждого водителя транспортного средства дата, время
		и показания одометра при выезде транспортного средства с парковки
		проставляются в путевом листе водителя транспортного средства, который
		первым выезжает с парковки, а дата, время и показания одометра при
		заезде транспортного средства на парковку - в путевом листе водителя
		транспортного средства, который последним заезжает на парковку.
	\item Даты и время проведения предрейсового и послерейсового медицинского
		осмотра водителя проставляются медицинским работником, проводившим
		соответствующий осмотр, и заверяются его подписью с указанием фамилии и
		инициалов.

		По результатам прохождения предрейсового медицинского осмотра на путевом
		листе проставляется отметка "прошел предрейсовый медицинский осмотр, к
		исполнению трудовых обязанностей допущен".

		По результатам прохождения послерейсового медицинского осмотра
		проставляется отметка "прошел послерейсовый медицинский осмотр".
	\item Даты и время выпуска транспортного средства на линию и его
		возвращения, а также проведения предрейсового или предсменного контроля
		технического состояния транспортного средства проставляются должностным
		лицом, ответственным за техническое состояние и эксплуатацию
		транспортных средств, с отметкой "выпуск на линию разрешен" и заверяются
		его подписью с указанием фамилии и инициалов.
	\item Собственники (владельцы) транспортных средств обязаны регистрировать
		оформленные путевые листы в журнале регистрации путевых листов (далее -
		журнал).

		Журнал ведется на бумажном носителе, страницы которого должны быть
		прошнурованы, пронумерованы, и (или) на электронном носителе. При
		ведении журнала в электронной форме предусматривается обязательная
		возможность печати страниц журнала на бумажном носителе.
	\item В случае ведения журнала в электронной форме внесенные в него сведения
		заверяются усиленной квалифицированной электронной подписью.
\end{enumerate}

\end{document}
