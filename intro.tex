% vim:ft=tex:ts=2:sw=0:et
% Author: Plump Albert (plumpalbert@gmail.com)
\tocchapter{Введение}

В современных условиях многие показатели эффективности, такие как
производительность труда, объем грузоперевозок и др., существенно зависят от
технического состояния автопарка.

В ходе работы автомобиля, техническое состояние его компонентов ухудшается, что
впоследствии может привести к поломке. Существует два способа обеспечения работоспособности
автомобилей в эксплуатации при наименьших суммарных материальных и трудовых
затратах и потерях времени: поддержание работоспособности, называемое
техническим обслуживанием, и восстановление работоспособности, называемое
ремонтом.

Поддержание автотранспорта в исправном состоянии достигается техническим
обслуживанием и ремонтом на основе рекомендаций и правил по обслуживанию и
ремонту автомобильной техники. Основной идеей является то, что техническое
обслуживание транспортных средств -- это профилактическая мера, которая
принудительно проводится по специальному графику, а ремонт осуществляется лишь
по мере необходимости.

Согласно действующему регламенту, техническое обслуживание делится на следующие
виды в соответствии с периодичностью, перечнем и объемом работ: ежедневное
(ЕО), первое (ТО-1), второе (ТО-2) и сезонное (СО) техническое обслуживание.
Каждому обслуживанию предшествует диагностика автомобиля, осуществляемая на
основании государственного стандарта ГОСТ 33997-2016
[\ref{ref:ГОСТ-требования-то}], в котором перечислены требования безопасности к
техническому состоянию и методы проверки.

Целью технического обслуживания (ТО) автомобиля является отдаление момента
поломки, а следовательно снижение убытков организации из-за простоев техники и
непредвиденных затрат на ее восстановление.

Разрабатываемая система нацелена на простоту и удобство развертывания в целевых
организациях. Для этого автоматизируется процессы, связанные с установкой,
настройкой и обновлением продукта. В настоящее время стандартом решения данной
проблемы является использование технологий контейнеризации Docker и настройки
удаленных конфигураций Ansible.
