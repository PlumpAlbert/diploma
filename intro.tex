% vim:ft=tex:ts=2:sw=0:et
% Author: Plump Albert (plumpalbert@gmail.com)
\tocchapter{Введение}

В современных условиях многие показатели эффективности, такие как
производительность руда, объем грузоперевозок и др., существенно зависят от
технического состояния автопарка.

В процессе эксплуатации автомобиля происходит изменение его технического
состояния и агрегатов, которое может привести к частичной или полной потере
работоспособности. Существует два способа обеспечения работоспособности
автомобилей в эксплуатации при наименьших суммарных материальных и трудовых
затратах и потерях времени: поддержание работоспособности, называемое
техническим обслуживанием, и восстановление работоспособности, называемое
ремонтом.

Основная цель технического обслуживания (ТО) автомобиля состоит в предупреждении
и отдалении момента достижения предельного состояния. Это обеспечивается,
во-первых, предупреждением возникновения отказа путем контроля и доведения
параметров технического состояния автомобилей (агрегата, механизма) до
номинальных или близких к ним значений; во-вторых, предупреждением с момента
наступления отказа в результате уменьшения интенсивности изменения параметра
технического состояния, снижения темпа изнашивания сопряженных деталей благодаря
проведению смазочных, регулировочных, крепежных и других работ. ТО-1 и ТО-2
производятся по достижении определенного пробега (в зависимости от типа и модели
транспортного средства ТО-1 -- через 2-4 тыс. км, ТО-2 -- 6-20 тыс. км). При
ТО-1 производятся диагностика и обслуживание узлов, обеспечивающих безопасность
движения, при ТО-2 -- диагностика и обслуживание элементов, обеспечивающих
тягово-экономические свойства автомобиля.

Основным назначением сезонного обслуживания (СО) является подготовка автомобилей
к эксплуатации в холодное и теплое время года. Для общих климатических условий
СО совмещается преимущественно с ТО-2 или ТО-1 при соответствующем увеличении
трудоемкости основного вида обслуживания.

Для проведения грамотного ТО и СО необходимо четкое соблюдение правил, описанных
в нормативно-технической документации (НТД). Но поскольку компоненты НТД
постоянно обновляются и корректируются, оперировать подобной документацией
значительно сложно. Одним из путей устранения указанных сложностей и
совершенствования организации ТО является применение компьютерных средств для
оперирования НТД в процессе непосредственного выполнения операций обслуживания.

Применение информационных технологий позволяет обрабатывать информационные
потоки и формировать информационную базу данных технической эксплуатации
автомобилей, включающую в себя информационно-справочные базы данных,
информационные системы диагностирования и прогнозирования, что является одним из
резервов повышения работоспособности машин.

Разрабатываемая система нацелена на простоту и удобство развертывания в целевых
организациях. Для этого автоматизируется процессы, связанные с установкой,
настройкой и обновлением продукта. В настоящее время стандартом решения данной
проблемы является использование технологий контейнеризации Docker и настройки
удаленных конфигураций Ansible.
