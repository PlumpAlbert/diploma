% vim:ft=tex:ts=2:sw=0:noet Author: Plump Albert (plumpalbert@gmail.com)
\tocchapter{Список литературы}

% магическая команда для написания эл. ресурсов
\makeatletter
\newcommand{\keril}[3][]{
	{#2}.
	[Электронный ресурс]{%
		\def\tmp{#1}%
		\ifx\tmp\empty%
		\else
			{: \tmp}%
		\fi%
	}. --
	Электронные данные. --
	Режим доступа: \url{#3}
	(дата обращения: 20.05.2022).%
}
\makeatother

\begin{enumerate}
	\item \href
		{https://docs.cntd.ru/document/1200146241}
		{
			ГОСТ 33997-2016. Колесные транспортные средства. Требования к безопасности
			в эксплуатации и методы проверки (с Поправкой). -- Официальное издание.
			М.: Стандартинформ, 2018. -- 75 с.
		}
		\label{ref:ГОСТ-требования-то}

\item \keril
		{1С:Предприятие 8. Автосервис. Возможности}
		{https://solutions.1c.ru/catalog/autoservice/features}%
		\label{ref:1С-автосервис}

	\item \keril
		{iDirector}
		{https://auto.a25.ru/}%
		\label{ref:iDirector}

	\item \keril
		{Техническое обслуживание}
		{http://абсолютпб.рф/техническое-обслуживание}
		\label{ref:техническое-обслуживание}

	\item \href
		{https://zabgu.ru/files/html_document/pdf_files/fixed/Metodicheskie_rekomendacii_NTTS/Osnovy_rabotosposobnosti_texnicheskix_sistem._Uchebnoe_posobie.pdf}
		{ Озорнин С.П. Основы работоспособности технических систем: учебное пособие.
		[Текст] / Озорнин С.П., С.Я.Березин, А.И.Федотов -- Чита: ЧитГТУ, 2003 --
		121 c } \label{ref:рабочее-состояние}

	\item \href
		{https://kat-9.mskobr.ru/files/2018/_i_remont_dlya_SPO_(red_2)_svodnaya.pdf}
		{
			Шишлов А. Н. Техническое обслуживание и ремонт автотранспорта:
			учебнопрактическое пособие для автомобильных колледжей. [Текст] /
			Шишлов А. Н., Лебедев С. В., Быховский М.Л., Прокофьев В.В. --
			М.: ГБПОУ КАТ №9, 2017. – 352 с. }
		\label{ref:му-ремонт-и-техническое-обслуживание}

	\item
		Живой англо-русский словарь по вычислительной технике,
		информационным технологиям и связи [Электронный ресурс]: терминологический
		словарь. / Ред. Дмитриев В.А. --
		Электронные данные --
		Режим доступа:
		\url{http://www.morepc.ru/dict/}
		(дата обращения: \today).%
		\label{ref:календарное-планирование}

	\item
		\href{http://rusautomobile.ru/wp-content/uploads/dop_materials/books/03.12.2015/6/KuznetsovESUpravlenietehnicheskojekspluatatsiejavtomobilej(MTransport1990).pdf}%
		{Кузнецов Е.С. Управление технической эксплуатацией автомобилей --
		2-е изд., перераб. и доп. [Текст] / Кузнецов Е.С.; Ред. Белоцерковская
		С.И. -- М.: Транспорт, 1990. -- 272 с.}%
		\label{ref:кузнецов}

	\item \keril[Википедия. Свободная энциклопедия]%
		{Концептуальная модель}
		{https://ru.wikipedia.org/wiki/Концептуальная_модель}%
		\label{ref:wiki-er-диаграммы}

	\item \href%
		{https://eduherald.ru/ru/article/view?id=20048}
		{
			Волошин В.А. Инфологическая модель данных:
			пример построения ER-диаграммы [Текст] / Волошин В.А., Шляхов В.Д.,
			Барышевский С.О. // Международный студенческий научный вестник. – 2020.
			– № 2.
		}
		\label{ref:er-диаграммы}
		\label{ref:last}

\end{enumerate}
