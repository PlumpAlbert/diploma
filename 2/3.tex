\documentclass[../nirs.tex]{subfiles}
\usepackage{tabularray}

\begin{document}
\section{Экспериментальное изучение предметной области. Построение эмпирических
математических моделей}

Рассмотрим пример реализации стратегии \rom{1}, для которого $d_\text{П} =
d_\text{И}$.

Так как теоретически отказ может произойти при любой сколь угодно малой
наработке, то стратегия \rom{1} реализуется, как правило, не в чистом, а в
смешанном виде: допускается определенная (малая) вероятность отказа $F$ или
риск, а периодичность ТО или предупредительного ремонта $l_p$ берется равной
$x_{min} < l_p < \overline{x}$.

При этом те отказы, которые возникли раньше окончания периода $l_p$ устраняются
по мере их возникновения, т.е. по стратегии \rom{2}. Стоимость устранения этих
заказов как при \rom{1}, так и при \rom{2} стратегии равна $c$. Обычно задается
допустимая вероятность отказа $F$ или требуемая вероятность безотказной работы
$R$. Средняя наработка, при которой будут устраняться эти отказы:

\begin{equation*}
    l`_p =
    \left.
    \int_{x_{min}}^{l_p} lf(l)\,dl
    \middle/
    \left( \int_{x_{min}}^{l_p} f(l)\,dl \right)
    \right.
\end{equation*}

Остальные агрегаты будут обслуживаться с установленной периодичностью $l_p$,
стоимостью $d = d_\text{И}$ и вероятностью данного события $R = \gamma$ (на
рисунке \ref{fig:pdf.b} $x_\gamma$ -- гамма-процентный ресурс).

Преимущества стратегии \rom{1} состоят в следующем -- может быть гарантирован
определенный уровень надежности работы изделия; так как изделие не переходит
границу предельного состояния, то разовые затраты на поддержание исправного
состояния $d$, как правило, ниже, чем при отказе; предупредительный характер
стратегии \rom{1} создает предпосылки, для плановой организации ТО и Р.

Недостатком такой реализации, что в среднем периодичность предупредительных
работ оказывается значительно меньшей, чем средняя наработка до отказа, т.е.
$l_p < \overline{x}$. Отношение $\beta_0 = l_p \overline{x}$ называется
коэффициентом рациональной периодичности. Кроме того, эта стратегия требует
ведения определенного учета, планирования работ и соблюдения производственной
дисциплины. Поэтому там, где неудовлетворительно поставлены учет и экономический
анализ, где не умеют ценить рабочую силу и технику, а также наблюдается
фактический дефицит деталей, априорное предпочтение отдается стратегии \rom{2},
которая в принципе не может обеспечить активного управления работоспособностью.

Отмеченный выше недостаток реализации стратегии \rom{1}, а именно $l_p <
\overline{x}$, может быть компенсирован за счет того, что $c \gg d$. Поэтому
целесообразность применения стратегии \rom{1} отделяется
экономико-вероятностными методами, учитывающими соотношения $c/d$ и $l_p
\overline{x} = \beta_0$, а также законами распределения этих величин.

При такой реализации стратегии \rom{1} удельные затраты определяются как
отношение средневзвешенной стоимости одной операции к средневзвешенной
наработке, т.е.

\begin{equation*}
    C^I_1 = \frac{cF + d_\Pi R}{l_p R + l`_p F}
\end{equation*}

Затем, дифференцируя $dC^I_1 / dl = 0$, определяют $l_{01}$, соответствующее
$C^I_{1 min}$. Если $C^I_{1 min} < C^{II}$, то предпочтительной является
стратегия \rom{1}. В противном случае целесообразно придерживаться стратегии
\rom{2}.

Данный подход к определению оптимальной периодичности профилактики является
универсальным и пригоден для определения оптимальных периодичностей ТО и
предупредительного ремонта, периодичности и целесообразности диагностики и т.д.

Для профилактической стратегии с предварительным контролем (\rom{1}-2)
удельная стоимость операции равна:

\begin{equation*}
    C^I_2 = \frac%
        {cF + \left( d_K + d_\text{И} \right) R_1 + d_K R_2}%
        {l`_p F + l_p R}
    =
    \frac{cF + R \left( d_K + d_\text{И} \right)}%
        {l`_p F + l_p R},
\end{equation*}
где $R_1$ -- вероятность выполнения контрольной и исполнительской частей
операций профилактики; $R_2$ -- вероятность выполнения только контрольной части
операции.

При $C^II > C^I_{1 min} > C^I_{2 min}$ целесообразна стратегия \rom{1},
выполняемая с предварительным контролем (\rom{1}-2), при этом назначается
определенная наработка автомобиля -- периодичность ТО $l_{02}$, при которой
проводится контроль параметров технического состояния $Y$ и принимается решение
о необходимости выполнения исполнительской части операции (см. рисунок
\ref{fig:pdf.c}). Как правило, $l_{01} \neq l_{02}$. При стратегии \rom{1}-2
контроль технического состояния может осуществляться как стационарными, так и
встроенными в изделие контрольно-диагностическими средствами.

Таким образом, при рассмотрении профилактической задачи в общем виде необходимо
сделать для каждого технического воздействия (или групп воздействий) выбор одной
из возможных стратегий и определить для нее рациональные режимы. При этом
необходимо учитывать, во-первых, стоимостные характеристики операций; во-вторых,
вариации случайных величин; в-третьих, характер реализации процесса изменения
параметра технического состояния; в-четвертых, возможные требования и
ограничения, например, по допустимым дельным затратам, безотказности в
межосмотровые периоды и т.д.

Основываясь на проведенных в НИИАТе, МАДИ и других организациях исследованиях,
разработаны графические и аналитические зависимости, позволяющие оперативно
оценить существование в данных условиях рациональной предупредительной
стратегии.

Так, существование рациональной предупредительной стратегии (\rom{1}-1)
определяется из условия

\begin{equation}
    \label{eq:strategy}
    \left[
        \frac%
            { 2 k_\Pi \upsilon_x }%
            { \left( 1 + \upsilon^2_x \right) \left( 1 - \upsilon_x \right) }
    \right]^{ \upsilon_x }
    -
    \frac%
        { k_\Pi \upsilon_x }%
        { \left( 1 - \upsilon_x \right) }
    -
    k_\Pi
    >
    0,
\end{equation}
где $k_\Pi = d/c; \upsilon_x$ -- коэффициент вариации наработки на отказ при
стратегии \rom{2}.

Рассмотрим пример для объекта, имеющего показатели, указанные в таблице
\ref{tab:experimental_data}.
\begin{longtblr}
[
	caption = { Экспериментальные данные },
	label = {tab:experimental_data},
]
{
	hlines, vlines,
	colspec = {C C},
    width = \textwidth,
	rowhead = 1,
	rowfoot = 0,
}
    Условное обозначение & Значение \\

    $c$ & 10.5 \\
    $d$ & 1.5 \\
    $k_\Pi$ & 0.14 \\
    $\overline{x}$ & 80 \\
    $\upsilon_x$ & 0.5
\end{longtblr}

Подставив в формулу (\ref{eq:strategy}) получаем:

\begin{equation*}
    \left[
        \frac%
            { 2 \cdot 0.14 \cdot 0.5 }%
            { \left( 1 + 0.5^2 \right) \left( 1 - 0.5 \right) }
    \right]^{ \frac{1}{2} }
    -
    \frac%
        { 0.14 \cdot 0.5 }%
        { \left( 1 - 0.5 \right) }
    -
    0.14
    >
    0,
\end{equation*}

т.е. рациональная периодичность существует и стратегия \rom{1}-1 целесообразна.
Оптимальная периодичность определяется из выражения для коэффициента оптимальной
периодичности:

\begin{equation}
    \label{eq:coefficient}
    \beta_0 = \frac{l_0}{\overline{x}} =
    \left[
        \frac%
            { 2 k_\Pi \upsilon_x }%
            {
                \left( 1 + \upsilon_x^2 \right)
                \left( 1 - \upsilon_x \right)
            }
    \right]^{\upsilon_x}.
\end{equation}

В примере имеем $\beta_0 = 0.47$, откуда выражаем оптимальную периодичность
$ l_0 = 0.47 * 80 = 37.6 $ тыс. км.

Экономико-вероятностный метод позволяет определить рациональную периодичность ТО
также исходя из заданного сокращения потока отказов между очередными ТО. При
наличии ограничений по безотказности коэффициент рациональной периодичности:

\begin{equation*}
    \beta_0 \leq \left[
            \frac%
                { k_\omega }%
                { 0.5 \left( \upsilon_x^2 + 1 \right) }
        \right]^%
        { \upsilon_x / \left( 1 - \upsilon_x \right) }
        \, \text{при} \, \upsilon_x < 1,
\end{equation*}
где $\displaystyle k_\omega = \frac{ \omega_I }{ \omega_{II} }$ -- коэффициент сокращения
параметра потока отказов.

Если в рассматриваемом примере задано сокращение параметра потока отказов при
использовании предупредительной стратегии, например, в 5 раз ($k_\omega = 0.2$),
то коэффициент рациональной периодичности определяется по формуле
(\ref{eq:coefficient}):

\begin{equation*}
    \beta_0 \leq \left[
            \frac%
                { 0.2 }%
                { 0.5 \left( 0.5^2 + 1 \right) }
        \right]^%
        { 0.5 / \left( 1 - 0.5 \right) }
    \leq 0.32.
\end{equation*}

Рациональная периодичность в этом случае равна $l_0 = \beta_0 \overline{x} =
0.32 \cdot 80 = \, = 25.6$ тыс. км.

\end{document}
