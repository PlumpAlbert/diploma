\documentclass[../nirs.tex]{subfiles}

\begin{document}
\section{Теоретическое изучение предметной области. Построение теоретических
математических моделей}

Применяемые системы технического обслуживания и ремонта массовых изделий
базируются на определенных стратегиях обеспечения работоспособности. Всю
возможную совокупность наиболее типичных отказов и неисправностей автомобиля
(400-700 в зависимости от конструкции и условий работы) можно подразделить на
две большие группы: профилактируемые и непрофилактируемые. К последним
относятся, во-первых, отказы и неисправности, которые невозможно заранее
предвидеть у конкретного автомобиля, т.е. внезапные; во-вторых, отказы и
неисправности, которые нецелесообразно предотвращать по экономическим или иным
критериям. Таких отказов и неисправностей у современных автомобилей около
27-39\% от общего числа. Для них действует стратегия \rom{2}, заключающаяся в
том, что они устраняются по мере возникновения, т.е. по потребности. Иногда ее
называют \textquote{стратегией ожидания ремонта}. Если в качестве целевой
функции принять затраты, то для стратегии \rom{2} удельные затраты на ремонт:
\begin{equation*}
    C^{\rom{2}} =
    c\,/\,\bar{x} =
    c : \int_{x_{min}}^{x_{max}} x f(x)\,dx,
\end{equation*}
где $\bar{x},\,x_{min} \,\text{и} \,x_{max}$ -- соответственно средняя,
минимальная и максимальная наработки на отказ; $c$ -- разовые затраты на
устранение отказа; $f(x)$ -- плотность вероятности наработки на отказ.

Преимуществом стратегии \rom{2} является простота реализации , основным
недостатком -- неопределенность состояния конкурентного изделия, которое может
отказать в любое время, а также трудность планирования и организации
технического обслуживания и ремонта парка. Для профилактируемой группы отказов и
неисправностей может применяться как стратегия поддержания (проведение
технического обслуживания), так и стратегия восстановления (ремонт)
работоспособности. Выделение из этой группы профилактируемых отказов и
неисправностей производится исходя из заданных критериев эффективности, например
обеспечения необходимых уровней безопасности движения, минимизации затрат ТО и
Р, повышения уровня работоспособности, сокращения расхода топлива и т.д., причем
критерии эффективности могут меняться исходя из конкретных условий и
ограничений.

Стратегия \rom{1} -- профилактическая, предусматривает предупреждение
значительной доли отказов и неисправностей данного наименования, восстановление
исходного или близкого к нему технического состояния изделия до того, как будет
достигнуто предельное состояние. Поэтому разовые затраты на одно воздействие на
поддержание работоспособности на одно воздействие на поддержание
работоспособности по стратегии \rom{1} ($d_{\text{п}}$), как правило значительно
ниже соответствующих затрат стратегии \rom{2} ($c$), т.е. $c \gg d_{\text{п}}$,
что и является основным источником эффективности профилактической стратегии. Эта
стратегия реализуется при предупредительном техническом обслуживании,
диагностике, предупредительных заменах некоторых деталей, узлов, механизмов и
т.д. При стратегии \rom{1} устанавливается наработка (периодичность ТО), при
которой изделию восстанавливают исходное или близкое к нему техническое
состояние.

Применяются два основных метода реализации стратегии \rom{1}: планирование
воздействий по наработке с доведением параметра технического состояния до нормы
(\rom{1} -- 1); планирование контроля параметра технического состояния по
наработке с доведением до нормы в зависимости от фактического и допустимого
значений параметра технического состояния (\rom{1} -- 2). Поэтому при стратегии
\rom{1} профилактическая операция в общем виде состоит из двух частей --
контрольной и исполнительской:
\begin{equation*}
    d_{\text{п}} = d_{\text{к}} + k d_{\text{и}}\,,
\end{equation*}
где $d_{\text{п}}$ -- стоимость ТО (профилактики); $d_{\text{к}}$ -- стоимость
контрольно-диагностической части операции ТО; $k$ -- коэффициент повторяемости
исполнительской части операции ТО; $d_{\text{н}}$ -- стоимость исполнительской
части операции ТО [\ref{ref:кузнецов}, стр. 170].

\end{document}
