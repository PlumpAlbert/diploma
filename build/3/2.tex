\documentclass[../nirs.tex]{subfiles}

\begin{document}
\section{Описание источников информации, входных сигналов и документов}
Основными источниками информации в разрабатываемой системе являются:
\begin{enumerate}
	\item Входные данные пользователей системы, полученные от администратора,
		при их регистрации в системе.
	\item Входные данные техники, полученные от старшего механика, при их
		регистрации в системе.
	\item Данные о цене за транспортные услуги, полученные от экономиста.
	\item Информация об условиях эксплуатации автомобильной техники, полученная
		из приказов Министерства транспорта РФ.
	\item Информация о заказе, полученного от клиента, содержащая наименование
		заказчика, время подачи техники, наименование и вес перевозимого груза,
		адреса погрузки и разгрузки.
	\item Информация о выполнении задания водителем, содержащая в себе номер
		поездки, адреса и даты выезда и прибытия, номера товарно-транспортных
		накладных перевезенного груза, а также, в случае необходимости, адрес
		перецепки прицепа и порожий пробег прицепа.
	\item Информация о расходе топлива, содержащая наименование топлива, условия
		использования техники, начальные и конечные значения топлива в баке
		автомобиля, количество залитого топлива.
	\item Информация о путевом листе, включающая дату выдачи, дату истечения
		срока действия путевого листа, запланированные даты выезда и возвращения
		техники, результаты проведения медицинского осмотра водителя, результаты
		проведения технического осмотра техники, начальные и конечные показания
		одометра.
\end{enumerate}
\end{document}
