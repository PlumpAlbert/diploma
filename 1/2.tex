% vim:ft=tex:ts=2:sw=0:et
% Author: Plump Albert (plumpalbert@gmail.com)
\section{Характеристика предметной области}
В настоящее время автомобильный парк страны пополняется автотранспортными
средствами новой конструкции, совершенствуется структура подвижного состава,
увеличивается численность дизельного парка, растет число транспортных средств
большой грузоподъемности. Темпы роста объемов перевозок и численность подвижного
состава растут и нуждаются в увеличении численности предприятий технического
обслуживания автотранспорта. Это вызывает необходимость исследования пути
ускорения научно-технического прогресса в отрасли, определить рациональные формы
и направления развития производственно-технической базы.

Для поддержания автомобилей в технически исправном состоянии необходимо
своевременное прохождение технического обслуживания ТО-1. Для этого создают
специальные пункты, в которых имеется всё необходимое оборудование для их
диагностики и ремонта. Диагностика осуществляется на основании государственного
стандарта ГОСТ 33997-2016 [\ref{ref:ГОСТ-требования-то}], в котором указаны
требования безопасности к техническому состоянию и методы проверки.

Из-за повышающегося количества автотранспортных средств загруженность пунктов
технического обслуживания стала велика. Существует несколько вариантов решения
данной проблемы. Возможно увеличение числа пунктов технического обслуживания.
Либо увеличить пропускную способность таких пунктов, за счет совершенствования
программы технического обслуживания ТО−1, улучшая ее технологичность,
качественность и респектабельность.

Нормативно-техническая документация (НТД), являющаяся одной из главных
составляющих системы технической эксплуатации машин, в том числе и грузовых
автомобилей, рассредоточена во многих публикациях. Определенные компоненты НТД,
в частности, по техническому обслуживанию (ТО) и техническому диагностированию
(ТД) грузовых автомобилей, постоянно развиваются и корректируются.
В силу указанных особенностей оперирование такой документацией,
включая подбор и систематизацию ее обновленных компонентов, представляет
значительные сложности, из-за чего на практике специалисты тратят много времени
на оперирование НТД и нередко используют устаревшие или неполные комплекты
документации. Это влечет к возможному снижению качества обслуживания
автотранспорта, что нередко случается на практике.
Одним из путей устранения указанных сложностей и совершенствования организации
ТО грузовых автомобилей является применение компьютерных средств для
оперирования НТД в процессе непосредственного выполнения операций обслуживания.
Однако в данной сфере отсутствуют доступные на практике приемы и методы
использования таких средств. Поэтому разработка приемов и методов ТО с
использованием указанных средств является одной из актуальных задач современной
инженерной науки.

Выполнение операций ТО может быть усовершенствовано применением соответствующих
технологических карт (ТК). Однако они разработаны лишь для некоторых моделей
автомобилей и не получили широкого практического применения.
Операции ТД весьма эффективны в процессе ТО. Однако из-за широкого спектра их
применимости в технической эксплуатации машин они пока не получили тесной
корреляции с операциями ТО, т.е. явно не вписаны в процессы ТО.

Еще одним приемом усовершенствования ТО является реализация метода
прогнозирования параметров состояния агрегатов и узлов автомобиля по результатам
ТД. Из-за значительной трудоемкости и сложности вычислений по прогнозированию,
отсутствия систематизированных данных по реализации, метод также не получил
практического применения.
