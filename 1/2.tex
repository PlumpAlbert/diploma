\documentclass[../nirs.tex]{subfiles}

\begin{document}
\section{Объекты управления, информационные объекты и автоматизируемые процессы.
	Пользователи и внешние сущности}
Объектом управления данной системы являются процессы учета и анализа
деятельности грузоперевозок предприятием. Деятельность данного предприятия
нацелено на удовлетворение потребностей в транспортировке большого количества
грузов на различные расстояния. Целевой аудиторией данной компании являются
преимущественно крупные организации, в частности мебельные и строительные
магазины.

Информационными объектами в данной системе служат путевые листы, информация о
заказах, информация о расходах, показания датчиков автомобиля.

Автоматизируемые процессы в данном проекте:
\begin{itemize}
	\item учет заказов - сбор данных о поступающих на предприятие заказах, а
		также расчет экономических показателей для конкретного заказа;
	\item учет расходов - сбор и хранение данных об издержках производственной
		деятельности предприятия;
	\item учет ремонтных и сервисных работ - анализ длительности работы
		автомобильного средства без проведения ТО, а также учет факта ремонтных
		работ и формирование отчетов по произведенным работам.
	\item анализ деятельности - сбор данных и формирование различных отчетов о
		деятельности организации.
\end{itemize}

Пользователями системы являются диспетчер, экономист, бухгалтер, механик ГСМ,
старший механик и водитель автомобиля.

Диспетчер принимает заявки от клиентов, формирует путевые листы,
фиксирует выдачу путевых листов водителям.

Экономист и бухгалтер формируют отчеты о деятельности организации.

Механик ГСМ ответственен за учет горюче-смазочных материалов.

Старший механик ответственен за техническое обслуживание автомобильной
техники, календарное планирование графика проведения ТО, выпуск
автомобиля на линию.

Водитель вводит информацию о перемещении транспорта.

Внешними сущностями выступают клиент и автозаправочная станция.

\end{document}
