\documentclass[../nirs.tex]{subfiles}

\begin{document}
\section{Объекты управления, информационные объекты и автоматизируемые процессы.
  Пользователи и внешние сущности}
Объектом управления разрабатываемой системы являются процессы учета и анализа
деятельности сервиса обслуживания автомобильной техники.
Его деятельность нацелена на поддержание парка автомобилей в технически
исправном состоянии.

Информационными объектами в разрабатываемой системе служат:
\begin{itemize}
	\item нормативно-техническая документация технического обслуживания
		транспортных средств;
	\item нормативно-техническая документация технической диагностики
		транспортных средств;
	\item формуляр автомобиля, содержащий в себе информацию о пробеге,
		ранее проведенных работах и т.д.
\end{itemize}

Автоматизируемые процессы в разрабатываемой системе:
\begin{itemize}
	\item учет ресурсов, необходимых для проведения ТО -- сбор и хранение данных
		о ресурсах, используемых при проведении технического обслуживания
		автомобильной техники;
	\item календарное планирование проведения ТО -- планирование технических
		работ для каждой единицы техники, с учетом текущей загруженности
		мастеров, располагаемыми ресурсами;
	\item учет проведения сервисных работ -- сбор и хранение информации о
		текущем статусе ТО для каждой единицы техники.
\end{itemize}

Пользователями системы являются старший техник и мастер.

Старший техник ответственен за регистрацию автомобильной техники в системе,
внесению данных нормативно-технической документации, календарное планирование
графика проведения ТО, распределение работ по ТО среди мастеров, учет доступных
материалов для проведения технического обслуживания.

Мастер ответственен за техническое обслуживание автомобильной техники.

Внешней сущностью системы является федеральное агентство по техническому
регулированию и метрологии (Росстандарт).

\end{document}
