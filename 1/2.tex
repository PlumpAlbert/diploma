\section{Объекты управления, информационные объекты и автоматизируемые процессы.
  Пользователи и внешние сущности}
Объектом управления разрабатываемой системы являются процессы учета и анализа
деятельности сервиса обслуживания автомобильной техники.
Его деятельность нацелена на поддержание парка автомобилей в технически
исправном состоянии.

Информационными объектами в разрабатываемой системе служат:
\begin{itemize}
	\item формуляр автомобиля, в котором содержится информация о ранее
        проведенных работах;
	\item данные о техническом обслуживании автомобиля, содержащие статус
        выполнения, дату начала, окончания, планируемые сроки выполнения;
	\item данные о выполненных работах, количестве затраченных ресурсов и
        их стоимости.
\end{itemize}

Автоматизируемые процессы в разрабатываемой системе:
\begin{itemize}
	\item учет ресурсов, необходимых для проведения ТО -- сбор и хранение данных
		о ресурсах, используемых при проведении технического обслуживания
		автомобильной техники;
	\item календарное планирование проведения ТО -- планирование технических
		работ для каждой единицы техники, с учетом текущей загруженности
		мастеров, располагаемыми ресурсами;
	\item учет проведения сервисных работ -- сбор и хранение информации о
		текущем статусе ТО для каждой единицы техники.
\end{itemize}

Пользователями системы являются старший техник и автослесарь.

Старший техник ответственен за регистрацию автомобильной техники в системе,
внесению данных нормативно-технической документации, календарное планирование
графика проведения ТО, распределение работ ТО среди автослесарей, учет доступных
материалов для проведения технического обслуживания.

Автослесарь ответственен за выполнение задач, выданных старшим техником.

Внешней сущностью системы может выступать федеральное агентство по техническому
регулированию и метрологии (Росстандарт), издающее стандарты в области
технического обслуживания и диагностики транспортных средств.
