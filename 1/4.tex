% vim:ft=tex:ts=2:sw=0:et
% Author: Plump Albert (plumpalbert@gmail.com)
\section{Порядок контроля и приемки}
Для системы устанавливаются следующие этапы испытаний:
\begin{itemize}
  \item предварительные испытания;
  \item опытная эксплуатация;
  \item приемочные испытания.
\end{itemize}

Представительные испытания Системы проводятся для определения ее
работоспособности и возможности приемки Системы в Опытную эксплуатацию.
Предварительные испытания организует Заказчик, и проводит их совместно с
Разработчиком.

В сводном Протоколе испытаний приводится заключение о возможности приемки системы
в Опытную эксплуатацию, а также перечень необходимых доработок и сроки их
выполнения. Работа завершается оформлением Акта приемки в Опытную эксплуатацию.

Продолжительность Опытной эксплуатации -- не менее двух месяцев. Во время
Опытной эксплуатации Системы ведут Рабочий журнал, в который заносят:
\begin{itemize}
  \item сведения о продолжительности функционирования Системы;
  \item сведения об отказах, сбоях, аварийных ситуациях;
  \item сведения об изменениях параметров объекта автоматизации;
  \item сведения о проведенных корректировках программного обеспечения и
    документации.
\end{itemize}

Приемочные испытания системы проводят для определения соответствия техническому
заданию и документации проекта. Приемочную комиссию образуют приказом по
предприятию. В состав комиссии входят представители заказчика и разработчика.
