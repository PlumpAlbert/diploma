% vim:ft=tex:ts=4:sw=0
% Author: Plump Albert (plumpalbert@gmail.com)
\section{Характеристика предприятия прохождения практики}
\subsection{Характеристика деятельности организации}
С начала основания в 2006 году компания \textquote{Fusionsoft} разрабатывает
программные модули, интегрируемые в различные информационные системы, мобильные
и Web- приложения. За долгое время существования у компании сформировался список
постоянных клиентов. Среди которых -- \textquote{NLMK DanSteel A/S} -- крупный
производитель толстого листа в Европе, который входит в группу компаний НЛМК. За
время сотрудничества с NLMK DanSteel были созданы следующие программные
продукты [\ref{ref:fs}]:
\begin{itemize}
    \item программное обеспечение автоматизации агрегата маркировки продукции;
    \item программное обеспечение претензионной работы;
    \item анализа и прогнозирования износа парка валов;
    \item и многие другие.
\end{itemize}

Также в 2019 году компания была выбрана подрядчиком для автоматизации агрегатов
резки и системы планирования и эффективности производства на крупнейшем
международном предприятии по производству стальной продукции -- НЛМК [\ref{ref:fs}].

Помимо группы компаний НЛМК основным клиентом \textquote{Fusionsoft} выступает
компания \textquote{65apps}, которая входит в Топ-4 мобильных разработчиков
России. Задачей компании является обеспечение качественного back-end --
безопасного, производительного и надежного -- к разрабатываемым заказчиком
мобильных приложений [\ref{ref:fs}].

\subsection{История библиотеки React.js}
React.js (иногда React или ReactJS) -- JavaScript-библиотека с открытым исходным
кодом для разработки пользовательских интерфейсов. Разрабатывается и
поддерживается Facebook, Instagram и сообществом отдельных разработчиков и
корпораций [\ref{ref:wiki-react}].

React был создан Джорданом Валке, разработчиком программного обеспечения из
Facebook. На него оказал влияние XHP -- компонентный HTML-фреймворк для PHP.
Впервые React использовался в новостной ленте Facebook в 2011 году и позже в
ленте Instagram в 2012 году. Исходный код React открыт в мае 2013 года на
конференции \textquote{JSConf US} [\ref{ref:wiki-react}].

React Native анонсирован на конференции Facebook \textquote{React.js Conf} в
феврале 2015 года, а исходный код открыт в марте 2015 года. Он позволяет
разрабатывать нативные Android-, iOS- и UWP-приложения с использованием React [\ref{ref:wiki-react}].

18 апреля 2017 года Facebook анонсировал React Fiber, переписанную и
оптимизированную версию React. React Fiber станет основой разработки всех
будущих функций и улучшений [\ref{ref:wiki-react}].

React может использоваться как для разработки одностраничных, так и для
мобильных приложений. Его цель -- предоставить высокую скорость, простоту и
масштабируемость. В качестве библиотеки для разработки пользовательских
интерфейсов React часто используется с другими библиотеками, такими как MobX,
Redux и GraphQL [\ref{ref:wiki-react}].

Ключевыми особенностями библиотеки React являются [\ref{ref:wiki-react}]:
\begin{itemize}
    \item однонаправленная передача данных -- свойства передаются от
        родительских компонентов к дочерним. Компоненты получают свойства как
        множество неизменяемых значений, поэтому компонент не может напрямую
        изменять свойства, но может вызывать изменения через callback-функции.
        Такой механизм называеют \textquote{свойства вниз, события наверх};

    \item виртуальный DOM -- React использует виртуальный DOM (англ. virtual
        DOM). React создаёт кэш-структуру в памяти, что позволяет вычислять
        разницу между предыдущим и текущим состояниями интерфейса для
        оптимального обновления DOM браузера. Таким образом программист может
        работать со страницей, считая, что она обновляется вся, но библиотека
        самостоятельно решает, какие компоненты страницы необходимо обновить;

    \item JSX -- JavaScript XML (JSX) -- это расширение синтаксиса JavaScript,
        которое позволяет использовать HTML-подобный синтаксис для описания
        структуры интерфейса. Как правило, компоненты написаны с использованием
        JSX, но также есть возможность использования обычного JavaScript.
\end{itemize}
