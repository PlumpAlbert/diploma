% vim:ft=tex:ts=4:sw=0
% Author: Plump Albert (plumpalbert@gmail.com)
\section{Характеристика предприятия прохождения практики}
\subsection{Характеристика деятельности организации}
С начала основания в 2006 году компания \textquote{Fusionsoft} разрабатывает
программные модули, интегрируемые в различные информационные системы, мобильные
и Web- приложения. За долгое время существования у компании сформировался список
постоянных клиентов. Среди которых -- \textquote{NLMK DanSteel A/S} -- крупный
производитель толстого листа в Европе, который входит в группу компаний НЛМК. За
время сотрудничества с NLMK DanSteel были созданы следующие программные
продукты [\ref{ref:fs}]:
\begin{itemize}
    \item программное обеспечение автоматизации агрегата маркировки продукции;
    \item программное обеспечение претензионной работы;
    \item анализа и прогнозирования износа парка валов;
    \item и многие другие.
\end{itemize}

Также в 2019 году компания была выбрана подрядчиком для автоматизации агрегатов
резки и системы планирования и эффективности производства на крупнейшем
международном предприятии по производству стальной продукции -- НЛМК [\ref{ref:fs}].

Помимо группы компаний НЛМК основным клиентом \textquote{Fusionsoft} выступает
компания \textquote{65apps}, которая входит в Топ-4 мобильных разработчиков
России. Задачей компании является обеспечение качественного back-end --
безопасного, производительного и надежного -- к разрабатываемым заказчиком
мобильных приложений [\ref{ref:fs}].

\subsection{Краткое описание и история системы управления конфигурациями Ansible}
Ansible -- система управления конфигурациями, написанная на языке
программирования Python, с использованием декларативного языка разметки YAML для
описания конфигураций. Используется для автоматизации настройки и развертывания
программного обеспечения. Обычно используется для управления Linux-узлами, но
Windows также поддерживается. Поддерживает работу с сетевыми устройствами, на
которых установлен Python версии 2.4 и выше по SSH или WinRM соединению
[\ref{ref:wiki-ansible}].

Автор платформы -- Michael DeHaan, ранее разработавший серверную систему
развертывания ПО Cobbler [\ref{ref:cobbler}] и соавтор фреймворка удаленного
администрирования Func. Система Ansible входит в состав большинства
дистрибутивов Linux. Есть пакеты для Solaris, FreeBSD и MacOS. Компания Ansible,
Inc осуществляла коммерческую поддержку и сопровождение Ansible. 16 октября 2015
года Red Hat, Inc [\ref{ref:redhat}] объявила о поглощении Ansible, Inc.

Наряду с Chef, Puppet и SaltStack считается одной из наиболее популярных систем
управления конфигурациями для Linux. Главное отличие Ansible от аналогов -- не
нужна установка агента/клиента на целевые системы.

Пользователь Ansible создаёт определённые \textquote{плейбуки} (англ. playbook,
play -- игра, пьеса, book -- книга) в формате YAML с описанием требуемых
состояний управляемой системы. \textquote{Плейбук} -- это описание состояния
ресурсов системы, в котором она должна находиться в конкретный момент времени,
включая установленные пакеты, запущенные службы, созданные файлы и многое
другое. Ansible проверяет, что каждый из ресурсов системы находится в ожидаемом
состоянии и пытается исправить состояние ресурса, если оно не соответствует
ожидаемому.

Для выполнения задач используется система модулей. Каждая задача представляет
собой имя задачи, используемый модуль и список параметров, характеризующих
задачу. Система Ansible поддерживает переменные, фильтры обработки переменных
(поддержка осуществляется библиотекой Jinja2 [\ref{ref:jinja2}]), условное
выполнение задач, параллелизацию, шаблоны файлов. Адреса и настройки целевых
систем содержатся в файлах \textquote{инвентаря} (inventory). Поддерживается
группирование конечных систем, для реализации набора сходных задач
[\ref{ref:wiki-ansible}].
