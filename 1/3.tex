% vim:ft=tex:ts=2:sw=0:et
% Author: Plump Albert (plumpalbert@gmail.com)
\section{Требования к системе}
\subsection{Внешняя среда}
Документы автомобилей, информация о пользователях и материалах.

\subsection{Функции системы}
Система должна выполнять следующие функции:
\begin{itemize}
  \item учет ресурсов, необходимых для проведения ТО;
  \item учет текущего состояния автомобилей;
  \item учет проведения сервисных работ;
  \item календарное планирование проведения ТО.
\end{itemize}

\subsection{Пользователи системы}
Пользователями системы являются старший техник и автослесарь.

Старший техник ответственен за регистрацию автомобильной техники в системе,
внесению данных нормативно-технической документации, календарное планирование
графика проведения ТО, распределение работ ТО среди автослесарей, учет доступных
материалов для проведения технического обслуживания.

Автослесарь ответственен за выполнение задач, выданных старшим техником.

\subsection{Входные и выходные данные}
Основными источниками информации в разрабатываемой системе являются:
\begin{enumerate}
  \item Входные данные пользователей системы, полученные от администратора,
    при их регистрации в системе.
    \item Входные данные техники, полученные от старшего техника, при их
    регистрации в системе.
  \item Информация о текущем количестве запасных материалов на складе.
  \item Информация об износе деталей автомобиля, во время прохождения им
        диагностики.
  \item Информация о необходимых материалах для проведения технического
        обслуживания и ремонта автомобиля.
\end{enumerate}

Выходной информацией в разрабатываемой системе является:
\begin{enumerate}
  \item График прохождения технического осмотра -- содержит календарный график
    прохождения технического осмотра каждой единицей техники.
  \item Отчеты по затратам -- содержат информацию о количестве расходов в
    рамках заданного пользователем временного промежутка.
  \item Планирование выполнения ремонтных работ с учетом имеющихся
    ресурсов.
\end{enumerate}


\subsection{Требования к аппаратной и программной платформе для установки}
Требования к программному обеспечению:
\begin{itemize}
  \item операционная система Debian 11;
  \item программное обеспечение для автоматизации развёртывания и управления
    приложениями в средах с поддержкой контейнеризации Docker.
\end{itemize}

Требования к аппаратному обеспечению:
\begin{itemize}
  \item тактовая частота процессора: не менее 2 GHz;
  \item оперативная память (ОЗУ): не менее 1 GB;
  \item запоминающее устройство (ЗУ): не менее 5 GB свободного дискового
    пространства.
\end{itemize}


\subsection{Требования к надежности}
Показатели надеждности Системы должны отвечать требованиям ГОСТ 24.701-86 ЕСС
АСУ \textquote{Надежность автоматизированных систем управления. Основные
положения}. Обеспечение необходимого уровня надежности требует проведения
специального комплекса работ, выполняемых на разных стадиях создания и
эксплуатации системы.

При решении вопросов обеспечения требуемого уровня надежности системы необходимо
учитывать следующие особенности:
\begin{enumerate}
  \item В работе системы участвуют различные виды обеспечения, в том числе и так
    называемый \textquote{человеческий фактор}, который может в существенной
    степени влиять на уровень надежности системы;
  \item В состав системы входит большое количество разнородных элементов
    (включая технологический и эксплуатационный персонал). При этом в
    выполнении одной функции системы обычно участвуют несколько различных
    элементов, а один и тот же элемент может участвовать в выполнении нескольких
    функций системы.
\end{enumerate}

Поэтому при решении вопросов, связанных с надежностью системы, количественное
описание, анализ, оценка и обеспечение надежности необходимо проводить по каждой
функции системы в отдельности. В обоснованных случаях необходимо использовать
анализ возможности возникновения в системе аварийных ситуаций, ведущих к
значительным техническим или экономическим потерям.

\subsection{Требования по эргономике и технической эстетике}
Интерфейс системы должен быть понятным и удобным, не должен быть перегружен
графическими элементами и должен обеспечивать быстро отображение экранных форм.
Навигационные элементы должны быть выполнены в удобной для пользователя форме.
Средства редактирования информации должны удовлетворять принятым соглашениям в
части использования функциональных клавиш, режимов работы, поиска, использования
оконной системы. Ввод-вывод данных системы, прием управляющих команд и
отображение результатов их исполнения должны выполняться в интерактивном режиме.
Интерфейс должен соответствовать современным эргономическим требованиям и
обеспечивать удобный доступ к основным функциям и операциям системы.
