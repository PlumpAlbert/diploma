\section{Цели разработки, функции системы, ограничения и критерии оценки
результатов}

Упрощение задачи поддержки автомобильного парка в технически исправном состоянии
можно достичь путем ведения единой информационной базы по всем машинам и
автоматизации процессов контроля технического состояния автомобилей, выполнения
технического обслуживания машин, планированию задач по выполнению технического
обслуживания.

Основными факторами успешности разрабатываемой системы являются качество
технического состояния автопарка и снижение затрат при проведении технического
обслуживания техники. Качество технического состояния зависит от своевременного
проведения технического обслуживания и полноты проведения всего объема
поставленных задач.

Функциями системы являются:
\begin{itemize}

	\item учет ресурсов, необходимых для проведения ТО;

    \item учет текущего статуса проведения технического обслуживания
        автомобилей;

    \item генерация отчетов о затратах за период времени, указываемый
        пользователем;

    \item учет активных задач для каждого пользователя;

    \item помощь в вводе информации о проведении технического обслуживания на
        технике.

\end{itemize}
