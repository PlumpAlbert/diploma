\section{Литературный и патентный обзор постановки подобных задач.
Анализ стандартных средств и существующих способов решения задачи}

Автомобильный транспорт занимает одно из ведущих мест среди других видов
транспорта. В современных реалиях объемы перевозок неуклонно растут, что
приводит к увеличению числа автомобилей. И поскольку для поддержания
автомобилей в технически исправном состоянии необходимо своевременное
прохождение технического обслуживания, возникает проблема простоя техники во
время ожидания очереди на обслуживание. Для решения указаной проблемы
необходимо увеличить скорость обслуживания отдельного автомобиля на
специализированных пунктах. Это вызывает потребность исследования пути
ускорения научно-технического прогресса в отрасли, определить рациональные
формы и направления развития производственно-технической базы.

В организациях эксплуатационное состояние транспортных средств обеспечивается
производственно-технической службой, которая несет ответственность за
своевременное и качественное техническое обслуживание и ремонт в соответствии с
установленными стандартами. Для этого необходимо грамотно организовать работу
ремонтного и обслуживающего персонала, соблюдать требования
нормативно-технической документации по техническому обслуживанию и ремонту.

\subsection{Примеры существующих программных продуктов на рынке}
1C:Предприятие. Автосервис.

\textquote{1С:Автосервис 8} – это отраслевое специализированное решение, предназначенное
для автоматизации управления и учета в автосервисах, станциях технического
обслуживания и автомойках.

Конфигурация \textquote{Автосервис} разработана на основе типовой конфигурации
\textquote{Управление нашей фирмой}, редакции 1.6 системы программ
\textquote{1С:Предприятие 8} с сохранением всех возможностей и механизмов
типового решения. Учитывает специфику предприятий авто бизнеса и обеспечивает
следующие возможности [\ref{ref:1С-автосервис}]:
\begin{itemize}
	\item ведение базы клиентов с регистрацией и хранением всей важной
		информации;
	\item тотальный контроль всех контактов с клиентами: входящие и
		исходящие звонки, электронные письма, встречи и прочее;
	\item предварительная запись на ремонт;
	\item анализ клиентской базы;
	\item использование справочников: модели автомобилей, нормочасы, виды
		ремонта и другие;
	\item регистрация и хранение номенклатуры товаров и услуг;
	\item учет движения денежных средств в кассе и на банковских счетах;
	\item учет рабочего времени сотрудников и расчет заработной платы;
	\item статистика, отчеты и другие показатели;
	\item облачное решение обеспечивает доступ к системе через интернет из любых
		браузеров.
	\\[\baselineskip]
\end{itemize}


iDirector

iDirector представляет собой современное узкоспециализированное решение для
\textquote{продвинутых} автосервисов. Сочетает в себе легкий понятный интерфейс
и мощные модули. Также имеет облачную версию, доступную даже с мобильных
устройств.

Помимо стандартных возможностей, таких как: управление клиентской базой, ведение
заказов, учет склада, управление персоналом, имеется и ряд дополнительных
возможностей [\ref{ref:iDirector}]:
\begin{itemize}
	\item интеграция с сайтом -– позволяет установить форму для оформления
		заказов на сайт автосервиса, а также добавить онлайн-консультанта;
	\item таймлайн -- сравнение изображения с камер наблюдения и сопоставление с
		фактическим занесением заказа в систему;
	\item резервирование –- автоматическое регулярное резервирование всех
		данных, и возможность восстановления системы к определенной дате.
	\\[\baselineskip]
\end{itemize}


\textquote{АвтоДилер} с модулем \textquote{Сервис}

Система «АвтоДилер» -- это специализированное программное обеспечение для
автобизнеса.

Система предназначена для автоматизации учета, планирования и анализа работы
любых предприятий: крупных и мелких автомастерских, автосалонов, магазинов
автозапчастей, автомоек, шиномонтажных мастерских и станций замены масла,
автостраховщиков.

Модуль \textquote{Сервис} предназначен для автоматизации учета работ в автосервисах и на
станциях технического обслуживания автомобилей. Система позволяет значительно
сократить время на оформление документов. При повторном обращении клиента, у
пользователя имеется вся история взаимоотношений с ним и не потребуется
\textquote{ползать} по многотомным архивам сервиса, для восстановления картины
ремонта автомобиля.

В программном комплексе существует возможность оформления необходимых документов
как для клиента, так и для внутренних операций.

Решение также имеет возможность вести учет и создавать отчеты по выработке.

Из минусов стоит выделить отсутствие облачной версии, что сегодня является
довольно весомым недостатком.
