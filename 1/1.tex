\documentclass[../nirs.tex]{subfiles}

\begin{document}
\section{Литературный и патентный обзор постановки подобных задач.
Анализ стандартных средств и существующих способов решения задачи}

% В наши дни очень перспективным и стремительно развивающимся видом предоставления
% услуг являются грузовые перевозки. Уже стало обыденностью заказывать продукты с
% доставкой на дом, несколько лет успешно функционируют интернет-магазины одежды и
% техники. Из-за постоянно возрастающих человеческих потребностей на рынок выходят
% новые фирмы, деятельность которых заключается в доставке грузов.
%
% Основной целью автоматизации является повышение качества бизнес-процесса и
% снижение его стоимости. На сегодняшний день существует огромный выбор
% программных продуктов, позволяющих автоматизировать функцию учета деятельности
% на предприятии. Однако большая часть из них ориентирована на широкий круг
% потребителей, что делает эти системы не пригодными для ведения учета на
% предприятиях со специфическими потребностями к учету хозяйственной деятельности,
% в частности, на малых предприятиях, занимающихся перевозкой грузов.

Нормативно-техническая документация (НТД), являющаяся одной из главных
составляющих системы технической эксплуатации машин, в том числе и грузовых
автомобилей, рассредоточена во многих публикациях. Определенные компоненты НТД,
в частности, по техническому обслуживанию (ТО) и техническому диагностированию
(ТД) грузовых автомобилей, постоянно развиваются и соответственно корректируются.
В силу указанных особенностей оперирование такой документацией,
включая подбор и систематизацию ее обновленных компонентов, представляет
значительные сложности, из-за чего на практике специалисты тратят много времени
на оперирование НТД и нередко используют устаревшие или неполные комплекты
документации.
В силу сложности и многогранности операций обслуживания грузовых автомобилей в
данных условиях, возможны и снижения качества их обслуживания, что нередко
проявляется на практике.
Одним из путей устранения указанных сложностей и совершенствования организации ТО грузовых
автомобилей является применение компьютерных средств для оперирования НТД в процессе
непосредственного выполнения операций обслуживания. Однако в данной сфере
отсутствуют практически доступные приемы и методы использования таких средств.
Поэтому разработка приемов и методов ТОГА с использованием указанных средств
является одной из актуальных задач современной инженерной науки.

Выполнение операций ТО может быть усовершенствовано применением соответствующих
технологических карт (ТК). Однако они разработаны лишь для некоторых моделей
автомобилей и не получили широкого практического применения.
Операции ТД весьма эффективны в процессе ТО. Однако из-за широкого спектра их
применимости в технической эксплуатации машин они пока не получили тесной
корреляции с операциями ТО, т.е. явно не вписаны в процессы ТО. Еще одним
приемом усовершенствования ТО является реализация метода прогнозирования
параметров состояния агрегатов и узлов автомобиля по результатам ТД. Из-за
значительной трудоемкости и сложности вычислений по прогнозированию, отсутствия
систематизированных данных по реализации метод также не получил практического
применения.

\end{document}
