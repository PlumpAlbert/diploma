%! tex program = xelatex
\documentclass{lstu-diploma}
\title{Разработка информационной системы поддержки процесса технического
обслуживания автомобильного парка}

\begin{document}

\maketitle
\chapter{Введение}
С давних пор, одной из первоочередных задач, решаемых на
предприятиях, становилась функция учета хозяйственной деятельности. На
сегодняшний день существует огромный выбор программных продуктов,
позволяющих автоматизировать функцию учета деятельности на
предприятии. Однако большая часть из них, ориентирована на широкий круг
потребителей, что делает эти системы не пригодными для ведения учета на
предприятиях со специфическими потребностями к учету хозяйственной
деятельности.
В частности, учет хозяйственной деятельности на малом предприятии
занимающегося перевозкой грузов, имеет свои потребности в хранении и
представлении данных. Большинство решений по учету хозяйственной
деятельности нацелено на крупные компании.
Однако в современной России, велика доля индивидуальных
предпринимателей. Индивидуальным предпринимателем считается
физическое лицо, зарегистрированное в установленном законом порядке и
осуществляющее предпринимательскую деятельность без образования
юридического лица. По данным ФНС на сегодняшний день в России,
зарегистрировано около 3,2 млн ИП, из них свыше 300 тыс. заняты в сфере
перевозки грузов.
Для частных предпринимателей, очень важно уметь грамотно
распорядится своим временем, иметь под рукой инструменты контроля
основных показателей как доходов, так и расходов, нужна ИС которая
анализирует все эти факторы и выводит итоговый результат о том была ли
деятельность продуктивная или же работа шла в убыток.
Учитывая все потребности данного сегмента потребительского рынка,
был сформирован ряд требований к ИС, она должна быть:
- мобильной, поскольку рабочим местом предпринимателя в
области грузоперевозок является кабина автомобиля;

\end{document}
