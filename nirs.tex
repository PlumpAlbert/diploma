%! tex program = xelatex
\documentclass{lstu-diploma}

\title{Разработка информационной системы поддержки технического обслуживания автомобильного парка}
\academicTitle{к.т.н., доцент}
\teacher{Назаркин О.А.}

\begin{document}

\maketitle

\tableofcontents

\chapter{Постановка задачи}

\section{Литературный и патентный обзор постановки подобных задач.
Анализ стандартных средств и существующих способов решения задачи}

В наши дни очень перспективным и стремительно развивающимся видом предоставления
услуг являются грузовые перевозки.
Уже стало обыденностью заказывать продукты с доставкой на дом, несколько лет
успешно функционируют интернет-магазины одежды и техники.
Из-за постоянно возрастающих человеческих потребностей на рынок выходят новые
фирмы, деятельность которых заключается в доставке грузов.

Одной из первоочередных задач, решаемых на предприятиях, является функция
учета хозяйственной деятельности.
Ведение документооборота на бумажных носителях не позволяют вести оперативного
учета работы организации и затрудняет доступ к информации, существенно замедляя
ее анализ.
На современном уровне развития информационных технологий данные проблемы
разрешимы с внедрением автоматизированной информационной системы.

Основной целью автоматизации является повышение качества бизнес-процесса и
снижение его стоимости.
На сегодняшний день существует огромный выбор программных продуктов, позволяющих
автоматизировать функцию учета деятельности на предприятии.
Однако большая часть из них ориентирована на широкий круг потребителей, что
делает эти системы не пригодными для ведения учета на предприятиях со
специфическими потребностями к учету хозяйственной деятельности,
в частности, на малых предприятиях, занимающихся перевозкой грузов.
Большинство решений по учету хозяйственной деятельности нацелено на крупные
компании.

Так, компания ТЭК ИНСОФТ разработало автоматизированную систему учета работы
автотранспорта и дорожно-строительной техники \textquote{АТЦ}. Это специализированное
программно-аппаратное решение, предназначенное для транспортных предприятий и
транспортных подразделений предприятий [1].
Основными возможностями данного программного комплекса являются полное
соответствие транспортному законодательству РФ и возможность создавать
исчерпывающие отчеты.
Внедрение данной системы позволит заметно ускорить работу диспетчера,
выпускающего технику на линию.
Комплекс позволяет автоматизировать процесс контроля за эксплуатацией
автотранспорта на основе обработанной информации по путевым листам, картам
технического состояния и заявкам, а так же для учет расхода ГСМ.
В состав \textquote{АТЦ} входят следующие автоматизированные рабочие места:

\begin{itemize}
	\item АРМ диспетчера;
	\item АРМ техника ГСМ;
	\item АРМ механика;
	\item АРМ бухгалтера и экономиста;
	\item АРМ начальника;
	\item АРМ охранника и механика КПП.
\end{itemize}
В настоящее время предприятиями негосударственных форм собственности выполняется
90\% перевозок грузов автомобильным транспортом. Это объясняется тем, что
автотранспортная отрасль является одной из наиболее доступных с точки зрения
приватизации собственности.

Однако в современной России, доля индивидуальных предпринимателей превышает долю
юридических лиц.
Индивидуальным предпринимателем считается физическое лицо, зарегистрированное в
установленном законом порядке и осуществляющее предпринимательскую деятельность
без образования юридического лица. По данным ФНС на сегодняшний день в России,
зарегистрировано около 3.6 млн ИП.

Для частных предпринимателей, очень важно уметь грамотно распорядится своим
временем, иметь под рукой инструменты контроля основных показателей как доходов,
так и расходов, нужна ИС которая анализирует все эти факторы и выводит итоговый
результат о том была ли деятельность продуктивная или же работа шла в убыток.

Учитывая все потребности данного сегмента потребительского рынка, был
сформирован ряд требований к ИС, она должна быть:
\begin{itemize}
    \item мобильной, поскольку рабочим местом предпринимателя в области
		грузоперевозок является кабина автомобиля;
    \item иметь легкий и удобный в обращении интерфейс;
    \item должна поддерживаться большей частью современных мобильных устройств;
    \item должна поддерживать инструменты анализа, с целью отражения информации
		о эффективности деятельности предприятия.
\end{itemize}

Основываясь на данных потребностях, было принято решение разработки мобильного
приложения, поскольку именно в таком виде, система будет максимально удобной для
конечного пользователя и сможет отвечать всем поставленным требованиям.

Транспортная политика государства в области грузоперевозки заключается в:
\begin{itemize}
    \item сохранении в собственности важнейших транспортных предприятий,
		коммуникаций и объектов, имеющих стратегическое значение и
		обеспечивающих безопасность функционирования транспортных систем;
	\item формировании необходимой законодательной и нормативной правовой базы
		для создания благоприятных условий функционирования и развитии
		цивилизованного рынка частных транспортных услуг;
	\item ликвидация убыточных и неперспективных государственных транспортных
		предприятий, их приватизации;
	\item повышении конкурентоспособности российских перевозчиков на внешнем и
		внутренних рынках транспортных услуг.
\end{itemize}

Трудно переоценить роль транспорта как в масштабах предприятия, так и в
масштабах страны. Это подтверждают Н.А. Троицкая и А.Б. Чубаков в своей книге,
приведу отрывок: «Транспорт – очень трудоемкая отрасль, в которой занято более
10\% работающих граждан страны. Транспортная отрасль потребляет 60\% мирового
производства жидких нефтепродуктов, 20\% стали, 80\% свинца, 70\% синтетических
каучуков, 40\% лакокрасочных изделий и др.

На транспортировке одновременно находится примерно 27-30 млн т различных грузов.

Затраты на перевозку продукции и погрузочно-разгрузочные работы могут составлять
в среднем 15-18\% от общей стоимости перевозимой продукции, но по отдельным видам
грузов могут быть значительно выше (например, при перевозке нефтепродуктов они
доходят до 40\%, строительных грузов – до 50\%, пищевых продуктов – до 25\%, а
сельскохозяйственной продукции до 100\% в связи с плохим качество дорог в
отдельных регионах)».

В настоящий момент, в России активно развивается информатизация общества, в
частности, на грузовые автомобили, в обязательном порядке, устанавливаются
датчики ГЛОНАСС, направленные на слежение за передвижением грузового
автотранспорта. Данные изменения могут стать хорошим инструментом для сбора и
анализа данных о загруженности дорог, о их качестве, а также многих других
глобальных показателях.

Сейчас данные устройства являются обязательным средством мониторинга
передвижения грузовиков весом свыше 12 тонн, с целью взимания платы. Однако в
будущем, данные с этих устройств смогут удобным инструментом для отслеживания
положения автомобиля в дороге, такие  системы уже были у организаций, которые
позаботились об этом. Однако учитывая то что теперь датчики будут у всех, это
может подтолкнуть разработчиков на разработку программного продукта для широкой
пользовательской аудитории.

По данным сайта Вестник-ГЛОНАСС: «Рынок мониторинга транспорта в России уходит
своими корнями в совсем недалекое прошлое. Еще 10 лет назад многие
компании-интеграторы работали с простейшими трекерами, сконструированными на
базе GSM-терминалов, а то и мобильных телефонов, и единственными
существенно-важными передаваемыми данными являлись спутниковые GPS-координаты.
Но для зарождения рынка этого было вполне достаточно.

Ничто не стоит на месте. Рынок развивается, растут запросы потребителей, а
вместе с ними расширяется спектр предлагаемого оборудования и сервисов.
Следующей ступенью развития транспортной телеметрии стала возможность
мониторинга топлива. Что логично, ведь топливо является наиболее затратной
составляющей в обслуживании транспорта. На сегодняшний день пользователи желают
получать более подробные данные о собственном автопарке, и современные
технологии позволяют это сделать. В данной статье мы постараемся познакомить вас
с актуальными тенденциями рынка Fleet Managmenet и решениями, которые позволяют
реализовать возросшие запросы клиентов».

Учет затрат на транспортном предприятии имеет свои особенности: «При учете
затрат, связанных с перевозками, внимания заслуживают следующие вещи:

\begin{itemize}
    \item используемые транспортные средства должны обязательно найти отражение
		в учете: в балансе, если они собственные или взяты в лизинг с учетом на
		балансе получателя, или за балансом, если они арендованы или получены в
		лизинг с учетом на балансе лизингодателя. Это позволит обоснованно
		принимать к учету все затраты по их содержанию;
	\item должна иметь место регистрация транспортных средств за перевозчиком:
		постоянная, если средства в собственности, или временная, если они
		арендованы или взяты в лизинг. Наличие этой регистрации (даже когда она
		временная) обязывает перевозчика к начислению и уплате транспортного
		налога;
	\item отнесение в затраты ГСМ, необходимых для работы транспортных средств,
		производится в соответствии с утвержденными нормами их списания. Эти
		нормы либо утверждены законодательно (и должны применяться для
		определенных отраслей), либо разрабатываются компанией самостоятельно.
		Это требует организации учета расхода ГСМ по каждому из транспортных
		средств и применения соответствующего алгоритма списания с отнесением
		излишков расхода в затраты, не уменьшающие базу по прибыли;
	\item безопасность работы автотранспорта в зимних условиях зависит от
		применения специальных шин, рассчитанных не на одну зиму. Необходимой
		станет организация не только неоднократной выдачи этих шин со склада, но
		и приема их на хранение на летний период с соответствующим отражением
		этих операций в учете.
	\item непременными составляющими затрат станут расходы на:
	
	\begin{itemize}
	    \item страхование транспортных средств, которое будет включаться в
			затраты частями в течение времени действия страхового полиса;
		\item периодический технический осмотр транспорта;
		\item проведение регулярного технического обслуживания;
		\item обязательные первичные (при приеме на работу) и предрейсовые
			медосмотры лиц, управляющих транспортными средствами;
		\item оплату специальных перерывов в работе, предназначенных для отдыха,
			для лиц, управляющих транспортом;
		\item услуги по погрузо-разгрузочным работам, взвешиванию груза, очистке
			транспортных средств;
		\item оплату проезда по платным дорогам, за въезд на территорию
			предприятий, хранение грузов, использование подъездных путей, подачу
			вагонов».
	\end{itemize}
\end{itemize}

\section{Объекты управления, информационные объекты и автоматизируемые процессы.Пользователи и внешние сущности}

\section{Цели разработки, функции системы,ограничения и критерии оценки результатов}

\chapter{Изучение и моделирование предметной области}
\section{Выявление основных понятий и процессов, их свойств и закономерностей. Построение ER-диаграммы предметной области}

\section{Теоретическое изучение предметной области. Построение теоретических математических моделей}

\section{Экспериментальное изучение предметной области. Построение эмпирических математических моделей}

\chapter{Разработка информационной базы для решения задачи}

\section{Построение концептуальной и физической модели данных}

\section{Описание источников информации, входных сигналов и документов}

\section{Описание выходной информации: сигналов, документов и видеокадров}

\end{document}
